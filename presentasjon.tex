\documentclass{beamer}
\usepackage{lmodern} % load a font with all the characters
\usepackage{tikz}
\usepackage[utf8]{inputenc}


%Information to be included in the title page:
\title{Exploring Negami's Conjecture}
\author{Håvard Utne Terland}

\usetheme{Madrid}



\begin{document}
	
	\frame{\titlepage}
	
	\begin{frame}
	\frametitle{A short introduction}
	Negami's conjecture is a conjecture in \textbf{topological graph theory}. In this presentation, all graphs are assummed to be finite and simple.
	
	\begin{block}{Topological graph theory}
		Topological graph theory is (more or less) the study of graphs drawn on surfaces. An interesting combination of discrete mathematics and topology.
	\end{block}

	\begin{example}
		For a graph $G$, what's the smallest number $g \geq 0$ such that $G$ can be drawn on the orientable surface of genus $g$? NP-Complete, but computable. One can say interesting things about given classes of graphs, i.e $K_n$.
	\end{example}
	
\end{frame}

\begin{frame}
\frametitle{Graph covers}
\begin{definition}
	A cover of a graph $G$ is a graph $H$ such that there is a surjective map $p:H \rightarrow G$ which is an isomorphism when restricted to the any set on the form $N_{H}(v) \cup {v}$ for $v \in H$. ("local isomorphism")
\end{definition}
\begin{theorem}
	For connected graphs, any cover of it has a well-defined \textit{fold-number}
\end{theorem}

\begin{proof}{}
	Let $G$ be connected, let $u,v$ be two vertices in $V(G)$ and let $H$ be a cover of $G$ with projection $p$. Pick a path between them $P = (u,v_1,v_2,v_3,\dots,v_n,v)$ in $G$ and consider $P \subseteq G$ a subgraph of the graph $G$. $p^{-1}(P)$ is a disjoint set of paths, and give bijection between $p^{-1}(u)$ and $p^{-1}(v)$. 
\end{proof}

\end{frame}

\begin{frame}

\frametitle{Planar covers}
Some non-planar graphs have covers that are planar! $K_5$ and $K_{3,3}$, for example.

\begin{block}{Remark}
	Not all graphs have planar covers, since planar graphs must have a vertex of degree maximum 5. Thus $K_n$ cannot have planar cover for $n > 6$.
\end{block}


\begin{figure}[h]
	\input{tfiler/k5cover.tikz}
	\caption{A 2-fold planar cover of $K_5$}	
\end{figure}

\end{frame}

\begin{frame}
\frametitle{The conjecture}
Observation: Given a graph $G$, if we can draw it on $\mathbb{R}P^2$ we call it projective-planar. A projective-planar graph has a $2$-fold planar cover of $G$. Simply lift via the 2-1 map $p:S^2 \rightarrow \mathbb{R}P^2$.
\begin{block}{Negami's Conjecture}
	A connected graph has a planar cover iff it is projective-planar.
\end{block}

\end{frame}

\begin{frame}
\frametitle{Generalizing the conjecture}
\begin{block}{Observation}
	Negami's conjecture $\Longleftrightarrow$ A graph has a projective-planar cover iff it can be drawn on projective plane.
\end{block}

\begin{block}{Conjecture (Hliněný)}
	A graph can be drawn on the Klein bottle iff it has a Klein-cover.
\end{block}

\begin{example}
	$K_7$ does not have a klein cover (Hliněný 1999)
\end{example}

\end{frame}


\begin{frame}
\frametitle{A question}
\begin{block}{Which graph families cover themself?}
	$\mathcal{C}(\mathcal{F}) = \mathcal{F}$ for which graph families $\mathcal{F}$? Can we say anything non-trivial about this in general? 
\end{block}
\end{frame}

\end{document}